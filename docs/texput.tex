\title{Compression in uncorrelated color domains}

\maketitle

\tableofcontents

\section{Description}

\subsection{Rate-control}
%{{{
When the transform is orthogonal, the quantization step of a subband should be inversely proportional to the subband gain.  
%}}}

\subsection{Components correlation}
%{{{

Usually, some part of the data in an image is
\href{https://en.wikipedia.org/wiki/Data_redundancy}{redundant} (can
be removed without loss of information). In the case of a color image
in the RGB domain, the three components of each pixel (each one
measuring the energy in a different band of the
\href{https://en.wikipedia.org/wiki/Visible_spectrum}{visible
  spectrum}) can be
\href{https://en.wikipedia.org/wiki/Correlation_and_dependence}{correlated}.
\href{https://vicente-gonzalez-ruiz.github.io/transform_coding/}{Transform
  coding} applied between the color compoents can concentrate the
information (energy) of the image in a small set of coefficients, that
after quantization and entropy coding, can be compressed more
efficiently.\footnote{For example, if the energy of a color subband is
low, quantization could completely makes zero such subband, but the
reconstruction of the image would be reasonable. The most part of
entropy codecs (and of course, PNG's DEFLATE) reach higher compression
ratios with sequences of zeros.} This
\href{https://github.com/Sistemas-Multimedia/Sistemas-Multimedia.github.io/blob/master/milestones/06-YUV_compression/color_redundancy.ipynb}{notebook}
quantifies the redundancy related to the color domain.

\begin{comment}
To estimate the
\href{https://en.wikipedia.org/wiki/Redundancy_(information_theory)}{redundancy}
we have basically two options:
\begin{enumerate}
\item Compute the
  \href{https://en.wikipedia.org/wiki/Entropy_(information_theory)}{0-order
    (memoryless source) entropy} of the signal: the higher the
  entropy, the lower the redudancy. In fact, if we suppose that the
  samples of the signal are uncorrelated, the 0-order entropy is an
  exact measure of the expected bit-rate achieved by an
  \href{https://en.wikipedia.org/wiki/Arithmetic_coding}{arithmetic
    encoder} (the most efficient entropy compressor). Unfortunately,
  the 0-order entropy is usually only a estimation of the redundancy,
  i.e., lower bit-rates can be achieved in practice after using a high-order
  decorrelation.
\item A better way is to use an
  \href{https://en.wikipedia.org/wiki/Data_compression}{lossless
    compressor}: the higher the length of the compressed file compared
  to the length of the original file, the lower the
  redundancy.\footnote{If the length of the compressed file is equal or
  larger than the length of the original file, then, for the compressor
  that we are using, there is not redundancy in the original
  representation.} Notice, however, that although this estimation is
  more accurate than the 0-order entropy, in general, it depends on the
  compressor (different algoritms can provide different
  estimations).
\end{enumerate}
\end{comment}

%}}}

\subsection{Color spaces}
%{{{

The color of a pixel depends on the
\href{https://en.wikipedia.org/wiki/Visible_spectrum}{frequency of the
  light that the pixel captures}. Such information can be represented
in a number of different encoding systems known as
\href{https://en.wikipedia.org/wiki/Color_space}{color spaces}. Among
all those systems, the RGB color space is the most used because it can
be obtained directly from the light signal using color
filters.\footnote{Specifically, a red filter, a green filter and a
blue filter.}

The RGB color model has evident physical advantages, but in general is
quite redundant. In this milestone we will analyze:
\begin{enumerate}
\item the
  \href{https://en.wikipedia.org/wiki/Discrete_cosine_transform}{DCT
    (Discrete Cosine Transform)} applied to the color domain, i.e.,
  the color-DCT space,
\item the \href{https://en.wikipedia.org/wiki/YCbCr}{$\text{YCrCb}$} space, and
\item the \href{https://en.wikipedia.org/wiki/YCoCg}{$\text{YCoCg}$} space,
\end{enumerate}
that are more efficient from a coding perspective. Besides,
$\text{YCrCb}$ and $\text{YCoCg}$ color spaces, which are
\emph{luma}-based (luminance-based\footnote{Luminance can be
considered as the intensity part of a viual stimuli.} color domains,
rely on the idea of separating the luminance coefficients (Y) from two
\emph{chroma} coefficients (red and blue in the case of
$\text{YCrCb}$, orange and green in the case of $\text{YCoCg}$). These
transformations have two main advantages~\cite{burger2016digital}:
\begin{enumerate}
\item Compatibility with legacy black and white systems is maintained
  while at the same time the bandwidth of the signal can be optimized
  by using different transmission bandwidths for the brightness and the
  chroma components.\footnote{Notice, however, that in the digital
  world bandwidth savings are equivalent to reduce the sampling rate.}
\item Since the \href{https://en.wikipedia.org/wiki/Visual_system}{HVS
  (Human Visual System)} is not able to perceive detail in the
  chrominance as well as it does in the luminance, the amount of
  information, and consequently
  \href{https://en.wikipedia.org/wiki/Sampling_(signal_processing)}{sampling
    rate}, used in the chrominance can be generally reduced to 1/4
  of the used for the luminance without a noticeable distortion (see
  Fig.~\ref{fig:san-diego_chroma_subsampled}). This
  \href{https://en.wikipedia.org/wiki/Bandwidth_(computing)}{fact is
    used when compressing} digital still images and is one of the
  reason why, for example, the
  \href{https://en.wikipedia.org/wiki/JPEG}{JPEG} image compressor
  converts RGB images to $\text{YCrCb}$.
\end{enumerate}

\begin{figure}
  \centering
  \png{san-diego_chroma_subsampled}{1000}
  \caption{Visual effect of chroma subsamplig in the YCrCb domain. See
    this
    \href{https://github.com/Sistemas-Multimedia/Sistemas-Multimedia.github.io/blob/master/milestones/06-YUV_compression/chroma_subsampling.ipynb}{notebook}.}
  \label{fig:san-diego_chroma_subsampled}
\end{figure}

%}}}

\section{References}

\renewcommand{\addcontentsline}[3]{}% Remove functionality of \addcontentsline
\bibliography{maths,data-compression,signal-processing,DWT,image-compression,image-processing}
