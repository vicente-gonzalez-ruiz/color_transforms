% Emacs, this is -*-latex-*-

\title{The $\text{YCrCb}$ Transform}
\maketitle

\tableofcontents

\section{The $\text{YCrCb}$ color domain}
$\text{YCrCb}$ is a \emph{luma}-based
(luminance-based\footnote{Luminance can be considered as the intensity
part of a viual stimuli.}) color domain. This domain rely on the idea
of separating the luminance coefficients (Y) from two \emph{chroma}
coefficients (red and blue in the case of $\text{YCrCb}$). After using
the transform, most of the energy will be concentrated in the luma
($\text{Y}$) subband.

The $\text{YCrCb}$ color model can be considered the digital version
of the \href{https://en.wikipedia.org/wiki/YUV}{$\text{YUV}$ (analog)
  color model}, and in some way can be used to maintain the
compatibility with legacy black and white systems is maintained while
at the same time the bandwidth of the signal can be optimized by using
different transmission bandwidths for the brightness and the chroma
components.\footnote{Notice, however, that in the digital world
bandwidth savings are equivalent to reduce the sampling
rate.}~\cite{burger2016digital}

The $\text{YCrCb}$ color domain has been used in JPEG to remove both,
statistical and visual redundancy.

\section{The $\text{RGB} \Leftrightarrow \text{YCrCb}$ transform}
%{{{ 

To convert a (color) pixel from the $\text{RGB}$ domain into the
\href{https://en.wikipedia.org/wiki/YCbCr}{$\text{YCrCb}$} color
domain, we use the $\text{RGB/YCrCb}$ (analysis)
transform~\cite{malvar2008lifting}
%Eq.~\ref{eq:YCrCb}
%\href{https://docs.opencv.org/3.4/de/d25/imgproc_color_conversions.html}{can
%  be also written as}
\begin{equation}
  \begin{array}{lcl}
    \text{Y}  & = & 0.299\text{R} + 0.587\text{G} + 0.114\text{B} \\
    \text{Cr} & = & 0.713(\text{R} - \text{Y}) + \delta \\
    \text{Cb} & = & 0.564(\text{B} - \text{Y}) + \delta,
  \end{array}
  \label{eq:alternative_YCrCb}
\end{equation}
where,
\begin{equation}
  \delta = \left\{
  \begin{array}{ll}
    128 & \text{for 8 bits (unsigned) images},\\
    32768 & \text{for 16 bits (unsigned) images},\\
    0.5 & \text{for floating point (}[0,1]\text{) images}
  \end{array}
  \right.
\end{equation}
is used to avoid negative coefficients. As it can be seen, $\text{Cr}$
and $\text{Cb}$ are scaled versions of $\text{R} - \text{Y}$ and
$\text{B} - \text{Y}$, so $\text{Cr}$ and $\text{Cb}$ can be
interpreted as measures of how much red and blue content in a pixel
differs from luma, respectively. Notice also that for a gray pixel,
$\text{R}=\text{G}=\text{B}=\text{Y}$, and so
$\text{Cr}=\text{Cb}=0$~\cite{malvar2008lifting}.

\begin{equation}
  \begin{bmatrix}
    \text{Y} \\
    \text{Cr} \\
    \text{Cb}
  \end{bmatrix}
  =
  \begin{bmatrix}
    0.299   &   0.587 & 0.114 \\ 
    0.5     & -0.4187 & -0.0813 \\
    -0.1687 & -0.3313 & 0.5
  \end{bmatrix}
  %\cdot
  \begin{bmatrix}
    \text{R} \\
    \text{G} \\
    \text{B}
  \end{bmatrix},
  \label{eq:YCrCb}
\end{equation}
where $\text{Y}$ is the luma and $\text{CrCb}$ are the chromas. The
main reason for such mapping is that the HVS is much less sensitive to
the high-frequency information in the
chromas~\cite{burger2016digital}. Thus, encoding systems such as
\href{https://en.wikipedia.org/wiki/JPEG}{JPEG} can
\href{https://en.wikipedia.org/wiki/Downsampling_(signal_processing)}{downsample}
the chromas (usually by 2:1 in the horizontal and vertical
directions), as well as increase their quantization step sizes with
respect to the luma, to achieve further
\href{https://en.wikipedia.org/wiki/Data_compression_ratio}{compression}
without noticeable visual distortion~\cite{malvar2008lifting}.

As it can be seen, considering that the $\text{RGB}$ values ranges
between $0$ and $255$ (and rounding to the nearest integer),
$0\le\text{Y}\le 255$, $0\le\text{Cr}\le 255$ and
$0\le\text{Cb}\le 255$, and therefore, the number of bits that are
necessary to represent each $\text{YCrCb}$ component is 8 (although we
must use floating point arithmetic to perform the transform).

Finally, notice that the $\text{YCrCb}$ transform is not orthogonal
because the analysis filters are not independent. This can be seen in
the Eq.~\eqref{eq:alternative_YCrCb}, where the $\text{Cr}$
coefficients depend on the coefficients of $\text{Y}$, and therefore,
there is a dependency between both
\href{https://en.wikipedia.org/wiki/Basis_(linear_algebra)}{basis},
and something similar happens for the $\text{Cb}$ subband. This can be
also easely checked: $$0.299*0.5 + 0.587*(-0.4187) + 0.114*(-0.0813) =
-0.1055451 \neq 0,$$ $$0.299*(-0.1687) + 0.587*(-0.3313) + 0.114*0.5 =
-0.1879144 \neq 0,~\mathrm{and}$$ $$0.5*(-0.1687) + (-0.4187)*(-0.3313
) + (-0.0813)*0.5 = 0.01371531 \neq 0.$$

%}}}

The inverse (synthesis) transform is defined by
\begin{comment}
\begin{equation}
  \begin{bmatrix}
    \text{R} \\
    \text{G} \\
    \text{B}
  \end{bmatrix}
  =
  \begin{bmatrix}
    1  &  1.402  & 0 \\ 
    1  &  -0.714  &  -0.344 \\ 
    1  & 0  & 1.772
  \end{bmatrix}
  %\cdot
  \begin{bmatrix}
    \text{Y} \\
    \text{Cr} \\
    \text{Cb}
  \end{bmatrix},
\end{equation}
or alternatively, by
\end{comment}
\begin{equation}
  \begin{array}{lcl}
    \text{R} & = & \text{Y} + 1.403(\text{Cr} - \delta) \\
    \text{G} & = & \text{Y} - 0.714(\text{Cr} - \delta) - 0.344(\text{Cb} - \delta)\\
    \text{B} & = & \text{Y} + 1.773(\text{Cb} - \delta).
  \end{array}
  \label{eq:iYCrCb}
\end{equation}
that in matrix form is
\begin{equation}
  \begin{bmatrix}
    \text{R} \\
    \text{G} \\
    \text{B}
  \end{bmatrix}
  =
  \begin{bmatrix}
    1  &  1.403  & 0 \\ 
    1  &  -0.714  &  -0.344 \\ 
    1  & 0  & 1.773
  \end{bmatrix}
  %\cdot
  \begin{bmatrix}
    \text{Y} \\
    \text{Cr}-\delta \\
    \text{Cb}-\delta
  \end{bmatrix},
\end{equation}

\section{Decorrelation with the $\text{YCrCb}$}
The $\text{YCrCb}$ transform uses three non-orthogonal
(interdependent) filters to generate three
subbands~\cite{vruiz__transform_coding} that we will denote by
$\text{Y}$, $\text{Cr}$, and $\text{Cb}$. Since in most of the images,
most of the energy is concentrated in the subband $\text{Y}$, the
transformed image improves its RD curve. This can be checked in the notebook
\href{https://github.com/vicente-gonzalez-ruiz/color_transforms/blob/main/docs/YCrCb/notebooks/YCrCb_over_RGB.ipynb}{Removing
  RGB redundancy with the YCrCb transform}.

\section{Optimal quantization in the $\text{YCrCb}$ domain}

Unfortunately, as it can be also seen in the notebook
\href{https://github.com/vicente-gonzalez-ruiz/color_transforms/blob/main/docs/YCrCb/notebooks/YCrCb_over_RGB.ipynb}{Removing
  RGB redundancy with the YCrCb transform}, the gains of the synthesis
filters depends on the amplitude of the coefficients (this is a direct
consequence of the interdependency of the filters). Therefore, when
one of the subbands is quantized, for example $\text{Y}$, this error
will be propagated to the other subbands. This also implies that the
quantization error generated in one subband only can be measured in
the $\text{RGB}$ domain.

These dificulties make impossible to select the
optimal\footnote{Remember that the optimal quantization step sizes
should select in each subband a RD point where the slope of the RD
curve in the transform domain is the same.} quantization step sizes
pattern in the transform domain, considering only the RD curves of
each subband. The only solution is to solve this problem by brute
force, quantizing in the transformed domain and testing the result in
the image domain for different quantization patterns.

\begin{comment}
\section{Quantization in the $\text{YCrCb}$ domain}
%{{{ 
The YCrCb transform is not orthogonal and the relative synthesis
filters gains depends on the energy of the inversely transformed
components. In this case, we can estimate the distortion generated by
the quantization of a color subband, always measured in the RGB
domain, if the rest of subbands are unquantized. This can be the
algorithm:
\begin{enumerate}
\item Variying the $\Delta$, estimate the RD curve for each YCrCb
  subband, keeping the other subbands unquantized. The distortion must
  be measured in the RGB domain.
\item Sort the RD points by their slope.
\item Apply progressively the combinations quantization steps. The
  distortion can be measured in both, the transform and the RGB
  domains.
\end{enumerate}
See the notebook
\href{https://github.com/vicente-gonzalez-ruiz/color_transforms/blob/main/docs/YCrCb/notebooks/YCrCb_over_RGB.ipynb}{Removing
  RGB redundancy with the YCrCb transform}.

Notice that the only alternative to this ``fast'' rate-control
algorithm is to perform a brute-force search of quantization steps
combinations.

The synthesis filters gains are important because the quantization
steps of each YCrCb component should be adjusted in order to
effectively provide the desired number of
\href{http://www.winlab.rutgers.edu/~crose/322_html/quantization.pdf}{bins}
(different dequantized values) in each component.

The synthesis filters generates a gain of $||1^2 + 1^2 + 1^2||_2^2=3$
(square of the Euclidean norm) for the $\text{Y}$ component,
$||1.402^2 + 0.714^2 + 0^2||_2^2=2.4754$ for the $\text{Cr}$
component, and $||0^2 + 0.344^2+ 1.772^2||_2^2=3.25832$ for the
$\text{Cb}$ component. So, when compressing an image through
quantization, the QSs should be modulated accordinly (the higher the
gain, the higher the quantization error, and therefore, the smaller
the QS should be).

The RGB/YCrCb
transform is not orthogonal\footnote{The RGB/YCrCb is not orthogonal
because, for example, as we can see in the
Eq.~\ref{eq:YCrCb_analysis}, the value of Cr depends on the value of
Y, and therefore, there is a dependency between both
\href{https://en.wikipedia.org/wiki/Basis_(linear_algebra)}{basis}.}
and therefore the contribution of each channel to the quality of the
reconstructed image $\tilde{X}$ are not additive (this can be seen in
this
\href{https://github.com/Sistemas-Multimedia/Sistemas-Multimedia.github.io/blob/master/study_guide/06-color_transform/performance.ipynb}{notebook}).
\end{comment}

\section{References}
%{{{ 
\renewcommand{\addcontentsline}[3]{}% Remove functionality of \addcontentsline
\bibliography{data_compression,signal_processing,image_compression,image_processing}
%}}}


\begin{comment}
\subsubsection{Quantization}
After analyzing the frame (representing it in the YCrCb domain), the
next natural step is quantization. Unfortunately, the RGB/YCrCb
transform is not orthogonal\footnote{The RGB/YCrCb is not orthogonal
because, for example, as we can see in the
Eq.~\ref{eq:YCrCb_analysis}, the value of Cr depends on the value of
Y, and therefore, there is a dependency between both
\href{https://en.wikipedia.org/wiki/Basis_(linear_algebra)}{basis}.}
and therefore the contribution of each channel to the quality of the
reconstructed image $\tilde{X}$ are not additive (this can be seen in
this
\href{https://github.com/Sistemas-Multimedia/Sistemas-Multimedia.github.io/blob/master/study_guide/06-color_transform/performance.ipynb}{notebook}). For
that reason, and only for the sake of simplicity, we will use
\begin{equation}
  \Delta_{\text{Y}} = \Delta_{\text{Cr}} = \Delta_{\text{Cb}}.
  \label{eq:simple_Q}
\end{equation}
\end{comment}

\begin{comment}
Lets suppose that we use a static
uniform dead-zone quantizer with QSs $\Delta_{\text{Y}}$,
$\Delta_{\text{Cr}}$, and $\Delta_{\text{Cb}}$, for the components Y,
Cr, and Cb, repectively. Lets suppose also that the RGB/YCrCb
transform is orthogonal\footnote{The RGB/YCrCb
  is not orthogonal because, for example, as we can see in the
  Eq.~\ref{eq:YCrCb_analysis}, the value of Cr depends on the value of
  Y, and therefore, there is a dependency between both
  \href{https://en.wikipedia.org/wiki/Basis_(linear_algebra)}{basis}.},
that is, the filters of the analysis transform are orthogonal (that is
the same that the columns of the synthesis transform are orthogonal),
in order to assume that
\begin{equation}
  \Delta_{\text{Y}} = \Delta_{\text{Cr}} = \Delta_{\text{Cb}}.
  \label{eq:simple_Q}
\end{equation}
is a good quantization pattern. However, 

--------------

and under the
assumption of that the RGB/YCbCr is an
\href{https://en.wikipedia.org/wiki/Orthogonality}{orthogonal}
transform and that each channel is compressed independently, the
optimal quantization of the channels should use $\Delta_{\text{Y}}$,
$\Delta_{\text{Cr}}$, and $\Delta_{\text{Cb}}$ so that
\begin{equation}
  \lambda_{\text{Y}} = \lambda_{\text{Cr}} = \lambda_{\text{Cb}},
  \label{eq:optimal_quantization}
\end{equation}
for a given total bit-rate $R$ (see this
\href{https://github.com/Sistemas-Multimedia/Sistemas-Multimedia.github.io/blob/master/study_guide/06-color_transform/performance.ipynb}{notebook})~\cite{vetterli1995wavelets,sayood2017introduction}. Therefore,
if all the channels have the same gain\footnote{The gain of a
transform can be determined computing the squared norm of the rows of
the synthesis transform (the synthesis filters).}, a quantization
strategy that should approximate Eq.~\ref{eq:optimal_quantization} is
to use
\begin{equation}
  \Delta_{\text{Y}} = \Delta_{\text{Cr}} = \Delta_{\text{Cb}}.
  \label{eq:simple_Q}
\end{equation}
When the gains are not the same, the quantization steps should be
divided\footnote{The squared norms measure the contribution of each
component to the energy of the pixel, and therefore, the higher the
contribution, the lower the $\Delta$.} by the channel gains, that for
the RGB/YCrCb transform are:
\begin{equation*}
  \begin{array}{rl}
    \text{Y}: & 1^2 + 1^2 + 1^2 = 3\\
    \text{Cr}: & 1.402^2 + 0.714^2 = 2.4754\\
    \text{Cb}: & 0.344^2 + 1.772^2 = 3.25832
  \end{array}
\end{equation*}

% Si un coeficiente tiene una gran ganancia es implica que su
% quantization también se dejará sentir más que si la ganancia es
% menor. Por tanto, tiene sentido usar un QS mayor en aquellos
% coeficientes de menor ganancia.

Unfortunately, the RGB/YCrCb transform is not orthogonal. This
means that the quantization noise introduced in one of the channel
will also affect to the rest of channels, which will degrade the RD
performance. The lack of orthogonality also reduces the effectivity of
the previous algorithm for determining the optimal quantization steps.

\end{comment}

\begin{comment}

After analyzing the frame (representing it in the YCrCb domain), the
next natural step is quantization. Supposing that we will use a static
uniform dead-zone quantizer with quantization steps
$\Delta_{\text{Y}}$, $\Delta_{\text{Cr}}$, and $\Delta_{\text{Cb}}$,
for the coefficients Y, Cr, and Cb, repectively, and supposing that
the contribution to the reconstruction of $X$ of one of the
coefficients is not influenced by the contribution of the rest of
coefficients (for this, both color spaces (RGB and YCrCb) should be
\href{https://en.wikipedia.org/wiki/Orthogonality}{orthogonal}), the
optimal quantization steps $\Delta^*_{\text{Y}}$,
$\Delta^*_{\text{Cr}}$, and $\Delta^*_{\text{Cb}}$, can be found using
a constant slope
(\href{https://en.wikipedia.org/wiki/Rate-distortion_theory}{RD}-$\lambda$)
quantization
strategy~\cite{vetterli1995wavelets,sayood2017introduction}.

As it can be seen in this
\href{https://github.com/Sistemas-Multimedia/Sistemas-Multimedia.github.io/blob/master/study_guide/06-color_transform/performance.ipynb}{notebook},
a RD (Rate-Distortion) curve is a 2D graph where we represent the
distortion generated by the quantization as a function of the bit-rate
of the quantization indexes. Thus, the closer the curve to the point
(0,0) of the graph, the better the performance of the encoding system
in RD terms. Now, if we suppose that each component (Y, Cr, and Cb) is
quantized and compressed independently, we can find the optimal
quantization steps, given a maximum target bit-rate $R^{\text{max}}$,
selecting them as
\begin{equation}
  \lambda_{\text{Y}} = \lambda_{\text{Cr}} = \lambda_{\text{Cb}},
\end{equation}
where $\lambda(R)$ is the slope of the RD curve for a given bit-rate
$R$, satisfiying also that
\begin{equation}
  R_{\text{Y}} + R_{\text{Cr}} + R_{\text{Cb}} \le R^{\text{max}}.
\end{equation}

Unfortunately, the RGB-to-YCrCb transform is not orthogonal (for
example, in Eq.~\ref{eq:YCrCb_analysis}, the value of Cr depends on the
value of Y, and therefore, there is a dependency between both
\href{https://en.wikipedia.org/wiki/Basis_(linear_algebra)}{basis})\footnote{This
can be also seen computing the
\href{https://en.wikipedia.org/wiki/Dot_product}{inner product} of the
basis functions of the analysis transform (only the inner product of
orthogonal vectors is 0). Thus, for example, the product of the basis
functions for Y and Cr is $0.299\times 0.5+0.587\times (-0.4187) +
0.144\times (-0.0813) = -0.1055451$.} and therefore neither the RGB
and the YCrCb spaces. This dificults the finding of
$\Delta^*_{\text{Y}}$, $\Delta^*_{\text{Cr}}$, and
$\Delta^*_{\text{Cb}}$ because the quantization error generated in one
of the components influences the quantization error of the rest of
components, and when this happens, we cannot use CS-RS-QS.

Anyway, as you can see in this
\href{https://github.com/Sistemas-Multimedia/Sistemas-Multimedia.github.io/blob/master/study_guide/06-color_transform/performance.ipynb}{notebook},
the use of the YCrCb color domain can be beneficial, even using a
simple quantization strategy such as
\begin{equation}
  \Delta_{\text{Y}} = \Delta_{\text{Cr}} = \Delta_{\text{Cb}}.
\end{equation}
As it can be seen, the RD curves can be improved for most bit-rates,
and therefore, it can be an interesting tool for removing the
intercomponent redundancy from a pure mathematical point of view.
\end{comment}
