% Emacs, this is -*-latex-*-

\title{The $\text{YCoCg}$ Transform}
\maketitle

\tableofcontents

\section{The $\text{YCoCg}$ color domain}
$\text{YCoCg}$ is a \emph{luma}-based
(luminance-based\footnote{Luminance can be considered as the intensity
part of a viual stimuli.} color domain. This domain rely on the idea
of separating the luminance coefficients (Y) from two \emph{chroma}
coefficients (orange and green in the case of $\text{YCoCg}$).

The $\text{YCoCg}$ color model is a succesor of th $\text{YCrCb}$ one,
where the transform filters has been redefined to be more efficient
(considering more and new images). Moreover, it can be also considered
a the digital version of the
\href{https://en.wikipedia.org/wiki/YUV}{$\text{YUV}$ (analog) color
  model}.

\section{Color decorrelation with the $\text{YCoCg}$ transform}
The $\text{YCoCg}$ is a near-orthogonal~\cite{vruiz__YCoCg}
transform~\cite{vruiz__transform_coding} that has been specifically
designed to accumulate energy in the $\text{Y}$ luma
subband~\cite{vruiz__image_IO}. The subbands $\text{Co}$ and
$\text{Cg}$ represents the croma (orange and green, respectively). The
size of the subbands is the same than the number of pixels in the
original image.

\section{Effects of the transform}
Basically the same than in the case of the DCT:
\begin{enumerate}
\item Smaller entropy (better RD curves).
\item High dynamic range (more points in the RD curves).
\end{enumerate}

\section{Chroma redundancy}

Humans do not perceive detail in the chrominance as well as in they
does in the
luminance.~\cite{vruiz__visual_redundancy,burger2016digital}. As in
$\text{YCrCb}$, the
\href{https://en.wikipedia.org/wiki/Sampling_(signal_processing)}{sampling
  rate} of the chroma is usually reduces to 1/4 without any noticeable
distortion. This feature is used in some of the last image and video
compression systems such as
\href{https://en.wikipedia.org/wiki/JPEG_XR#Description}{JPEG XR} and
\href{https://en.wikipedia.org/wiki/High_Efficiency_Video_Coding#Video_coding_layer}{HEVC}.

\section{The $\text{RGB} \Leftrightarrow \text{YCoCg}$ transform}
%{{{ 

Clearly, orthogonality is a desired property in lossy compression
systems because it helps to isolate\footnote{Without considering the
rest of components.} the impact (in the case of a color transform) of
each color subband on the quality of the reconstruction of the image,
simplifying significantly the determination of the pattern of QSs that
generate the optimal RD curve.

Moreover, Eqs.~\ref{eq:YCrCb} and \ref{eq:iYCrCb} were derived by
\href{https://en.wikipedia.org/wiki/Principal_component_analysis}{Principal
  Component Analysis (PCA)} on old\footnote{Recorded with the first
analog color cameras in the 70's.} video data. The same procedure has
been carried out with newer\footnote{\cite{malvar2008lifting} is dated
in 2008.} images, obtaining
\begin{equation}
  \begin{bmatrix}
    \text{Y} \\
    \text{C}_1 \\
    \text{C}_2
  \end{bmatrix}
  =
  \begin{bmatrix}
    \frac{1}{3} & \frac{1}{3} &  \frac{1}{3} \\ 
    \frac{1}{2} &           0 & -\frac{1}{2} \\
   -\frac{1}{4} & \frac{1}{2} & -\frac{1}{4}
  \end{bmatrix}
  %\cdot
  \begin{bmatrix}
    \text{R} \\
    \text{G} \\
    \text{B}
  \end{bmatrix}
  \Leftrightarrow
  \begin{bmatrix}
    \text{R} \\
    \text{G} \\
    \text{B}
  \end{bmatrix}
  =
  \begin{bmatrix}
    1  &  1  & -\frac{2}{3} \\ 
    1  &  0  &  \frac{4}{3} \\ 
    1  & -1  & -\frac{2}{3}
  \end{bmatrix}
  %\cdot
  \begin{bmatrix}
    \text{Y} \\
    \text{C}_1 \\
    \text{C}_2
  \end{bmatrix}.
  \label{eq:optimal}
\end{equation}

The \href{https://en.wikipedia.org/wiki/YCoCg}{YCoCg} color transform
is orthogonal and the synthesis filters gains\footnote{The gain of a
transform can be determined computing the squared norm of the rows of
the synthesis transform (the synthesis filters).} are 22/9 (for
$\text{Y}$), 25/9 (for $\text{C}_1$) and 22/9 (for $\text{C}_2$).

Unfortunately, from a perceptual perspective we must impose (thinking
of the subsampling of the chromas) some features in a color transform
(such as the influence of the green channel on the luma channel should
be high) that violates the orthogonality constrain
\cite{malvar2008lifting}. For this reason the authors finally propose
the transform
\begin{equation}
  \begin{bmatrix}
    \text{Y} \\
    \text{Co} \\
    \text{Cg}
  \end{bmatrix}
  =
  \begin{bmatrix}
    \frac{1}{4} &  \frac{1}{2}  &  \frac{1}{4} \\ 
    \frac{1}{2} &            0  & -\frac{1}{2} \\
    -\frac{1}{4} &  \frac{1}{2}  & -\frac{1}{4}
  \end{bmatrix}
  %\cdot
  \begin{bmatrix}
    \text{R} \\
    \text{G} \\
    \text{B}
  \end{bmatrix}
  \Leftrightarrow
  \begin{bmatrix}
    \text{R} \\
    \text{G} \\
    \text{B}
  \end{bmatrix}
  =
  \begin{bmatrix}
    1  &  1  & -1 \\ 
    1  &  0  &  1 \\ 
    1  & -1  & -1
  \end{bmatrix}
  %\cdot
  \begin{bmatrix}
    \text{Y} \\
    \text{Co} \\
    \text{Cg}
  \end{bmatrix},
\end{equation}
that is near orthogonal\footnote{For example, $\frac{1}{4}\frac{-1}{4}
+ \frac{1}{2}\frac{1}{2} + \frac{1}{4}\frac{-1}{4} = \frac{1}{8}$, and
we should obtain always 0.}, and has channel gains 3 ($\text{Y}$), 2
($\text{Co}$) and 3 ($\text{Cg}$). %These gains suggest that

Again, notice that if the $\text{RGB}$ values ranges between $0$
and $255$ (and rounding to the nearest integer), $0\le\text{Y}\le 255$,
$-128\le\text{Co}\le 127$ and $-128\le\text{Cg}\le 127$, and
therefore, the number of bits that are necessary to represent each
component is $8$. Therefore, we can use the same QS range for each
component. See this \href{https://github.com/Sistemas-Multimedia/Sistemas-Multimedia.github.io/blob/master/milestones/06-YUV_compression/YCrCb_matrix.ipynb}{notebook}.

% }}}

\section{Quantization in the $\text{YCoCg}$ domain}
Unlike the DCT, the gains of the synthesis filters of the inverse
$\text{YCoCg}$ transform are not the same. As it can be seen in
\href{https://github.com/Sistemas-Multimedia/Sistemas-Multimedia.github.io/blob/master/contents/YCoCg_SQ/YCoCG_SQ.ipynb}{YCoCG\_SQ.ipynb},
the $\text{Y}$ and the $\text{Cg}$ subbands are $3/2$ times more
energetic than the $\text{Co}$ subband. This basically means that the
quantization error~\cite{vruiz__signal_quantization} generated in the
$\text{Y}$ and $\text{Cg}$ subbands is more relevant that the
quantization error generated in the $\text{Co}$ subband (exactly,
$1.5$ times more important). Therefore, the 

Again, like in the color-DCT, ignoring the possible effects of the
entropy encoding stage (that could compress more some color subbands),
the previous gains suggest to use
\begin{equation}
  \frac{3}{2}\Delta_{\text{Y}} = \Delta_{\text{Co}} = \frac{3}{2}\Delta_{\text{Cg}}.
\end{equation}
See this \href{https://github.com/Sistemas-Multimedia/Sistemas-Multimedia.github.io/blob/master/milestones/06-YUV_compression/YCrCb_matrix.ipynb}{notebook}.

If the RD slope of each point depends also on the performance of DEFLATE (something that is normal), we can find the optimal RD curve with the algorithm:
\begin{enumerate}
\item Varilling the $\Delta$, estimate (remember that YCoCg is only  near-orthonal), the RD curve of each YCoCg subband can de found (without considering the rest of subbands).
\item Sort the RD points by their slope.
\item Apply progressively the combinations of quantization
  steps. Measure the distortion in the RGB domain (in the transform
  domain could be only estimated).
\end{enumerate}

%}}}


\subsection{Rate-control}
%{{{
When the transform is orthogonal, the quantization step of a subband should be inversely proportional to the subband gain.  
%}}}

\section{Resources}
%{{{ 
\renewcommand{\addcontentsline}[3]{}% Remove functionality of \addcontentsline
\bibliography{data-compression,signal-processing,maths,DWT,image-compression,image-processing}
%}}}
