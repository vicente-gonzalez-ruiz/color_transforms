\title{Color Transformations for Compression}

\maketitle

\tableofcontents

\section{$\text{RGB}$ components correlation}
%{{{

Usually, some part of the data in an image is statistically
\href{https://en.wikipedia.org/wiki/Data_redundancy}{redundant} (can
be removed without loss of information). In the case of a color image
in the RGB domain, the three components of each pixel (each one
measuring the energy in a different band of the
\href{https://en.wikipedia.org/wiki/Visible_spectrum}{visible
  spectrum}) can be
\href{https://en.wikipedia.org/wiki/Correlation_and_dependence}{correlated}.
\href{https://vicente-gonzalez-ruiz.github.io/transform_coding/}{Transform
  coding} applied between the color components can concentrate the
information (energy) of the image in a small set of coefficients, that
after quantization and entropy coding, can be compressed more
efficiently.\footnote{For example, if the energy of a color subband is
low, quantization could completely makes zero such subband, but the
reconstruction of the image would be reasonable, at least from a pure
rate/distortion point of view. Moreover, the most part of entropy codecs
(including, for example, PNG's DEFLATE) reach higher compression
ratios with sequences of zeros.} The notebook
\href{https://github.com/vicente-gonzalez-ruiz/color_transforms/blob/main/docs/color_redundancy.ipynb}{Spectral (color) Redundancy}
quantifies the redundancy related to the color domain.

%}}}

\subsection{Color spaces for coding}
%{{{

The color of a pixel depends on the
\href{https://en.wikipedia.org/wiki/Visible_spectrum}{frequency of the
  light that the pixel captures}. Such information can be represented
in a number of different encoding systems known as
\href{https://en.wikipedia.org/wiki/Color_space}{color spaces}. Among
all those systems, the RGB color space is the most used because it can
be obtained directly from the light signal using color
filters.\footnote{Specifically, a red filter, a green filter and a
blue filter.}

The RGB color model has evident physical advantages, but in general is
quite redundant. Other color spaces such as the obtained using the
transforms
\begin{enumerate}
\item \href{https://vicente-gonzalez-ruiz.github.io/color_transforms/docs/3DCT}{DCT
    (Discrete Cosine Transform)} (applied to the color domain),
\item \href{https://vicente-gonzalez-ruiz.github.io/color_transforms/docs/YCrCb}{$\text{YCrCb}$} transform, and
\item \href{https://vicente-gonzalez-ruiz.github.io/color_transforms/docs/YCoCg}{$\text{YCoCg}$} transform,
\end{enumerate}
are more efficient from a coding perspective. Besides, $\text{YCrCb}$
and $\text{YCoCg}$ color spaces, which are ``\emph{luma}-based''
(luminance-based\footnote{Luminance can be considered as the intensity
part of a viual stimuli.} color domains, rely on the idea of
separating the luminance coefficients ($\text{Y}$) from two
\emph{chroma} coefficients (red and blue in the case of
$\text{YCrCb}$, and orange and green in the case of $\text{YCoCg}$).

In general, these three transformations Compatibility with legacy
black and white systems is maintained while at the same time the
bandwidth of the signal can be optimized by using different
transmission bandwidths for the brightness and the chroma
components~\cite{burger2016digital}.\footnote{Notice, however, that in
the digital world bandwidth savings are equivalent to reduce the
sampling rate.}

%}}}

\begin{comment}
\subsection{Rate-control and Compression in uncorrelated color domains}
%{{{

Rate-control can be performed through quantization in the transform
domain. Since the quantization error is generated in this domain, the gain of the filters used 

In general, the 

then the transform is orthogonal, the quantization step of a subband
should be inversely proportional to the subband gain.

%}}}
\end{comment}

\section{References}

\renewcommand{\addcontentsline}[3]{}% Remove functionality of \addcontentsline
\bibliography{maths,data_compression,signal_processing,DWT,image_compression,image_processing}
