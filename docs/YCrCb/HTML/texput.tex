% Emacs, this is -*-latex-*-

\title{The $\text{YCrCb}$ Transform}
\maketitle

\tableofcontents

\section{The $\text{YCrCb}$ color domain}
$\text{YCrCb}$ is a \emph{luma}-based
(luminance-based\footnote{Luminance can be considered as the intensity
part of a viual stimuli.} color domain. This domain rely on the idea
of separating the luminance coefficients (Y) from two \emph{chroma}
coefficients (red and blue in the case of $\text{YCrCb}$). After using
the transform, most of the energy will be concentrated in the luma
($\text{Y}$) subband.

The $\text{YCrCb}$ color model can be considered the digital version
of the \href{https://en.wikipedia.org/wiki/YUV}{$\text{YUV}$ (analog)
  color model}, and in some way can be used to maintain the
compatibility with legacy black and white systems is maintained while
at the same time the bandwidth of the signal can be optimized by using
different transmission bandwidths for the brightness and the chroma
components.\footnote{Notice, however, that in the digital world
bandwidth savings are equivalent to reduce the sampling
rate.}~\cite{burger2016digital}

\section{The $\text{RGB} \Leftrightarrow \text{YCrCb}$ transform}
%{{{ 

To convert a (color) pixel from the $\text{RGB}$ domain into the
\href{https://en.wikipedia.org/wiki/YCbCr}{$\text{YCrCb}$} color
domain, we use the $\text{RGB/YCrCb}$ (analysis)
transform~\cite{malvar2008lifting}
%Eq.~\ref{eq:YCrCb}
%\href{https://docs.opencv.org/3.4/de/d25/imgproc_color_conversions.html}{can
%  be also written as}
\begin{equation}
  \begin{array}{lcl}
    \text{Y}  & = & 0.299\text{R} + 0.587\text{G} + 0.114\text{B} \\
    \text{Cr} & = & 0.713(\text{R} - \text{Y}) + \delta \\
    \text{Cb} & = & 0.564(\text{B} - \text{Y}) + \delta,
  \end{array}
  \label{eq:alternative_YCrCb}
\end{equation}
where,
\begin{equation}
  \delta = \left\{
  \begin{array}{ll}
    128 & \text{for 8 bits (unsigned) images},\\
    32768 & \text{for 16 bits (unsigned) images},\\
    0.5 & \text{for floating point (}[0,1]\text{) images}
  \end{array}
  \right.
\end{equation}
is used to avoid negative coefficients. As it can be seen, $\text{Cr}$
and $\text{Cb}$ are scaled versions of $\text{R} - \text{Y}$ and
$\text{B} - \text{Y}$, so $\text{Cr}$ and $\text{Cb}$ can be
interpreted as measures of how much red and blue content in a pixel
differs from luma, respectively. Notice also that for a gray pixel,
$\text{R}=\text{G}=\text{B}=\text{Y}$, and so
$\text{Cr}=\text{Cb}=0$~\cite{malvar2008lifting}.

As it can be seen, considering that the $\text{RGB}$ values ranges
between $0$ and $255$ (and rounding to the nearest integer),
$0\le\text{Y}\le 255$, $0\le\text{Cr}\le 255$ and
$0\le\text{Cb}\le 255$, and therefore, the number of bits that are
necessary to represent each $\text{YCrCb}$ component is 8 (although we
must use floating point arithmetic to perform the transform).

Finally, notice that the $\text{YCrCb}$ transform is not orthogonal
because the analysis filters are not independent. This can be seen in
the Eq.~\ref{eq:alternative_YCrCb}, where the $\text{Cr}$ coefficients
depend on the coefficients of $\text{Y}$, and therefore, there is a
dependency between both
\href{https://en.wikipedia.org/wiki/Basis_(linear_algebra)}{basis},
and something similar happens for the $\text{Cb}$ subband. This can
be also easely checked: $$0.299*0.5 + 0.587*(-0.4187) +
0.114*(-0.0813) = -0.1055451 \neq 0,$$ $$0.299*(-0.1687) +
0.587*(-0.3313) + 0.114*0.5 = -0.1879144 \neq
0,~\mathrm{and}$$ $$0.5*(-0.1687) + (-0.4187)*(-0.3313 ) +
(-0.0813)*0.5 = 0.01371531 \neq 0.$$

%}}}

\section{Decorrelation with the $\text{YCrCb}$}
The $\text{YCrCb}$ transform uses three non-orthogonal
(interdependent) filters to generate three
subbands~\cite{vruiz__transform_coding} that we will denote by
$\text{Y}$, $\text{Cr}$, and $\text{Cb}$~\cite{vruiz__YCrCb}. Since in
most of the images, most of the energy is concentrated in the subband
$\text{Y}$, the transformed image improves its RD curve. This can be
checked in the
\href{https://github.com/Sistemas-Multimedia/Sistemas-Multimedia.github.io/blob/master/contents/YCrCb_SQ/YCrCb_SQ.ipynb}{notebook}.

As it can be also seen in this notebook, the $\text{YCrCb}$ transform
the gains of the synthesis filters depends on the amplitude of the
coefficients. This basically means that we cannot optimze the
quantization steps to minimize the distortion.

The $\text{YCrCb}$ color domain has been used in JPEG to remove both,
statistical and visual redundancy.

\section{Quantization in the $\text{YCrCb}$ domain}
%{{{ 
The YCrCb transform is not orthogonal and the relative synthesis
filters gains depends on the energy of the inversely transformed
components. In this case, we can estimate the distortion generated by
the quantization of a color subband, always measured in the RGB
domain, if the rest of subbands are unquantized. This can be the
algorithm:
\begin{enumerate}
\item Variying the $\Delta$, estimate the RD curve for each YCrCb
  subband, keeping the other subbands unquantized. The distortion must
  be measures in the RGB domain.
\item Sort the RD points by their slope.
\item Apply progressively the combinations quantization steps. The
  distortion can be measure in both, the color-DCT and the RGB
  domains.
\end{enumerate}
See this \href{https://github.com/Sistemas-Multimedia/Sistemas-Multimedia.github.io/blob/master/study_guide/06-color_transform/YCrCb_compression.ipynb}{notebook}.

Notice that the only alternative to this ``fast'' rate-control
algorithm is to perform a brute-force search of quantization steps
combinations.

The synthesis filters gains are important because the quantization
steps of each YCrCb component should be adjusted in order to
effectively provide the desired number of
\href{http://www.winlab.rutgers.edu/~crose/322_html/quantization.pdf}{bins}
(different dequantized values) in each component.

The synthesis filters generates a gain of $||1^2 + 1^2 + 1^2||_2^2=3$
(square of the Euclidean norm) for the $\text{Y}$ component,
$||1.402^2 + 0.714^2 + 0^2||_2^2=2.4754$ for the $\text{Cr}$
component, and $||0^2 + 0.344^2+ 1.772^2||_2^2=3.25832$ for the
$\text{Cb}$ component. So, when compressing an image through
quantization, the QSs should be modulated accordinly (the higher the
gain, the higher the quantization error, and therefore, the smaller
the QS should be).

The RGB/YCrCb
transform is not orthogonal\footnote{The RGB/YCrCb is not orthogonal
because, for example, as we can see in the
Eq.~\ref{eq:YCrCb_analysis}, the value of Cr depends on the value of
Y, and therefore, there is a dependency between both
\href{https://en.wikipedia.org/wiki/Basis_(linear_algebra)}{basis}.}
and therefore the contribution of each channel to the quality of the
reconstructed image $\tilde{X}$ are not additive (this can be seen in
this
\href{https://github.com/Sistemas-Multimedia/Sistemas-Multimedia.github.io/blob/master/study_guide/06-color_transform/performance.ipynb}{notebook}).

\section{References}
%{{{ 
\renewcommand{\addcontentsline}[3]{}% Remove functionality of \addcontentsline
\bibliography{data-compression,signal-processing}
%}}}
